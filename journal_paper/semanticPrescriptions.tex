
\documentclass[journal]{IEEEtran}


\usepackage{hyperref}
\usepackage{amsmath}
\usepackage{amssymb}
\usepackage{hhline} % easy to manage table borders
\usepackage{colortbl} % colored cells in tables
\usepackage{multirow}
\usepackage{enumerate}
\usepackage{slashbox} % cell with slash inside
\usepackage{makeidx}  % allows index generation
\usepackage[utf8]{inputenc}
\usepackage{graphicx}
\usepackage{setspace}

\newtheorem{definition}{Definition}
\newtheorem{example}{Example}

\def\sectionautorefname{Section}
\def\subsectionautorefname{Section}


\begin{document}
%
% paper title
% can use linebreaks \\ within to get better formatting as desired
\title{Semantic Prescriptions -- Towards Intelligent and Interoperable Medical Prescriptions }

\author{Ali~Khalili
        and~Bita~Sedaghati
\IEEEcompsocitemizethanks{\IEEEcompsocthanksitem A. Khalili is with the Institute of Informatics, University of Leipzig, Germany.
% note need leading \protect in front of \\ to get a newline within \thanks as
% \\ is fragile and will error, could use \hfil\break instead.
E-mail: khalili@informatik.uni-leipzig.de
\IEEEcompsocthanksitem B. Sedaghati is with Institute of Pharmacy, University of Leipzig, Germany.
E-mail: bita.sedaghati@uni-leipzig.de
}
}


% The paper headers
\markboth{International Journal On Advances in Life Sciences v 5 n 3\&4 2013}%
{Khalili \MakeLowercase{\textit{et al.}}: Semantic Medical Prescriptions}
% The only time the second header will appear is for the odd numbered pages
% after the title page when using the twoside option.
%

% make the title area
\maketitle


\begin{abstract}
%\boldmath
The recent proliferation of Linked Open Data that enables the integration of multiple disparate data sources brings into the spotlight a new generation of knowledge management applications.
Particularly in the domain of pharmaceutical research and development, many efforts have been done to create a linked open drug data.
In this paper we present the Pharmer as an approach to facilitate the creation of semantic prescriptions.
Semantic prescriptions are intelligent e-prescription documents enriched by drug-related meta-data thereby know about their content and the possible interactions.
In an e-health system, semantic prescriptions provide an interoperable interface which helps patients, physicians, pharmacists, researchers and companies to collaboratively improve the quality of pharmaceutical services by facilitating the process of shared decision making.
Pharmer provides different views for the different personas involved in the process of e-prescribing.
It employs datasets such as DBpedia, DrugBank, DailyMed and RxNorm to automatically detect the drugs in the prescription and to collect multidimensional data on them.
Eventually it warns of the possible drug interactions in the prescription.
We evaluate the feasibility of the Pharmer by conducting a usability evaluation and report on the quantitative and qualitative results of our survey.
\end{abstract}


% Note that keywords are not normally used for peerreview papers.
\begin{IEEEkeywords}
 Semantic prescription, e-prescription, semantic annotation, e-health.
\end{IEEEkeywords}



\IEEEpeerreviewmaketitle


\section{Introduction}
\label{intro}

\section{LODD Applications}
\label{lod}

\section{Semantic Content Authoring}
\label{sca}


\section{Semantic E-Prescribing}
\label{sep}

\subsection{Architecture}

\subsection{Features}

\section{Mobile App}
\label{mobile}

\section{Pharmer as a professional network}
\label{pharmernet}
\paragraph{Pharmer offers couple of advantages over current e-prescribing systems:
The main benefit of using semantic prescriptions is the persistent connection to up-to-date drug information coming from multiple dynamic data sources.
So, when a change occurs to a drug (e.g. change in its effects or interactions) the semantic prescription automatically adopts to this new change.
Once writing a prescription it is very critical to consider drug interactions.
Drug interactions are divided to three categories namely \emph{food-drug}, \emph{drug-drug} and \emph{drug-plant} interactions.
Coadministration can either be synergistic or antagonistic which respectively increase or decrease the drugs effect.
The interactions may sometimes lead to change in the drug effect.
By applying semantic prescriptions, all types of drug interactions are prevented and the probability of errors in prescriptions are reduced to a great extend.
A semantic prescription is a self-contained document which is aware of its content and is connected to the linked open data.
In contrast to database-oriented e-prescriptions, semantic prescriptions can easily be exchanged among other e-health systems without need to changing their related infrastructure hence enabling a connection between physicians, pharmacists, patients, pharmaceutical researchers, insurance and drug companies.
Furthermore, semantic prescriptions increase the awareness of patients.
They provide patients with all the related information of the prescribed drugs thereby mitigating the possible misuse of drugs.
In addition, semantic prescriptions support shared decision making (SDM) by allowing patients and service providers to make health care decisions together.
Pharmer as a prescribing tool is able to be incorporated in a health care social network.
Such a network composed of health care professionals and patients who collaboratively write, correct and modify prescriptions in a semantically enriched environment.
As information source, the network access LODD, where diagnostic and prescribing data has been located.
Privacy of that network is also a critical point worth considerations.
\subsection{Shared Decision Making}
\label{subsec: SDM}
The traditional model of medical decision-making, in which doctors make decisions on treatment has no longer used in updated health care.The role of the patient, instead, in the consultation has been highlighted, mainly through introducing‘patient-centred’ strategies. Therefor, nowadays the models promoting patients' active involvement in the decision-making procedure becoming developed.
 A model introduced by Charles et al. defines shared decision making only under the following four key characteristics.
 These keys are:
 \begin{enumerate}
   \item both the patient and the doctor are involved
   \item both parties share information
   \item both parties take steps to build a consensus about the preferred treatment
   \item an agreement is reached on the treatment to implement
 \end{enumerate}

Pharmer as social network facilitates shared decision making through the connection amongst patient, physician and pharmacist. According to Charles et al. model, pharmer not only connects ptients nd physicians but also pharmacist as third party has an supervisory role on medication choice.

\subsection{Fast Diagnostic Tool}

\subsection{Privacy}

\section{Example Scenario}
\label{es}

\section{Outlook}
\label{outlook}

\section{Conclusion}
\label{sec:conclusion}

\begin{spacing}{0.3}
%\bibliographystyle{IEEEtran}
%\bibliography{refs}
\end{spacing}

% that's all folks
\end{document}


